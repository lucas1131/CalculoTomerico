\documentclass[12pt,a4paper]{article}

\usepackage[top=2.5cm,bottom=2cm,left=2cm,right=2cm]{geometry} 
\usepackage[brazil]{babel}
\usepackage{blindtext}
\usepackage{ragged2e}
\usepackage{soul}
\usepackage[utf8]{inputenc}
\usepackage{indentfirst}
\usepackage{mathtools}
\usepackage{amsmath}
\setlength\parindent{24pt}

%capa
\usepackage{multicol}
\usepackage{multirow}
\usepackage{graphicx}
\usepackage{float}
\usepackage{setspace} % espaçamento
\usepackage{tabularx}
\usepackage{booktabs}
\usepackage{array}

\newcommand{\university}{Universidade de São Paulo - ICMC}
\newcommand{\discipline}{SME0104 - Cálculo Numérico}
\newcommand{\data}{$1^o$ semestre / 2018}
\newcommand{\teacher}{Prof. Murilo Francisco Tomé}
\newcommand{\PAE}{PAEs: Gabriel Spadon de Souza, Paulo Henrique de Oliveira}
\newcommand{\specification}{Método de Gauss-Seidel para aproximação de soluções de sistemas não lineares}

\newcommand{\members}{
    \begin{table}[!ht]
        \centering
        \begin{tabular}{ll}
            \large\textsc{Letícia Rina Sakurai} & \large\textsc{Nº USP: 9278010} \\
            \large\textsc{Lucas Alexandre Soares} & \large\textsc{Nº USP: 9293265} \\
            \large\textsc{Matheus Henrique Soares} & \large\textsc{Nº USP: 8066349} \\
        \end{tabular}
    \end{table}{}
}

\newcommand{\capaicmc}{
    \begin{center}
        \begin{center}
            \begin{table}[!ht]
                \centering 
                \begin{tabular}{cl}
                    \multirow{4}{*}{\includegraphics[height=1.8cm,keepaspectratio=true]{logo-header.png}}
                    & \university\\
                    %& \course\\ 
                    & \discipline\\
                    & \teacher\\
                    %& \PAE\\
                \end{tabular}
            \end{table}
        \end{center}
        
        \vfill
        
        {\huge \specification}
        
        \vfill
        
        \doublespacing
        \large\textsc{\members}
        
        \vfill
        
        \large São Carlos - SP \\
        \large \today \\

        \end{center}
        
        \newpage
}

\begin{document}

% Capa
\capaicmc

\tableofcontents
\newpage

% Seta espaçamento entre linhas
\singlespacing

% começa finalmente 
% Aqui explica como este documento está estruturado e por que ele está assim, qual o motivo de usarmos este doc

%Explicação do problema e do método numérico
\section{Introdução}
O objetivo do trabalho proposto é a implementação do Método Iterativo de Gauss-Seidel, também conhecido como o Método de Liebmann, para a resolução de sistemas lineares $Ax = b$.  
\par O Método de Gauss-Seidel utiliza da decomposição LU, que divide a matriz A em uma matriz triangular inferior L e uma matriz traingular superior estrita U. A partir dessa decomposição podemos reescrever o sistema linear como $Lx = b - Ux$, ou na forma analítica: \[x^{(k+1)} = L^{-1}(b-Ux^{(k)})\]
\par Podemos aproveitar a forma triangular de L e então calcular os elementos de $x^(k+1)$ sequencialmente utilizando a seguinte substituição, até chegarmos ao critério de parada, que pode ser um determinado valor de erro:

\[x_{i}^{(k+1)} = \frac{1}{a_{ii}} \left ( b_{i} - \sum_{j=1}^{i-1} a_{ij}x_{j}^{(k+1)} - \sum_{j=1}^{n} a_{ij}x_{j}^{(k)}\right ),\>\>\>\> i = 1, 2, ... , n,\>\>\>\> k = 0, 1, 2, ...\]

%Correção e detalhamento   
\section{Resultados}
O teste do programa elaborado para resolução do sistema linear foi feito com dois casos, sendo $n = 50, 100$, \(b_{i} = \sum_{j=1}^{n} a_{ij}\), $i=1,...,n$ e $itmax = 1000$.
\par Para $n=50$, o resultado obtido foi:
\par Para $n=100$, o resultado obtido foi:


\begin{table}[!ht]
    \centering
    \begin{tabular}{llllll}
        Result = (& 0.999317, & 0.998988, & 0.998715, & 0.998294, & 0.997934,
        \\ & 0.997598, & 0.997234, & 0.996881, & 0.996539, & 0.996194,
        \\ & 0.995855, & 0.995521, & 0.995192, & 0.994867, & 0.994549,
        \\ & 0.994236, & 0.993929, & 0.993628, & 0.993334, & 0.993046,
        \\ & 0.992766, & 0.992493, & 0.992228, & 0.991971, & 0.991721,
        \\ & 0.991480, & 0.991248, & 0.991024, & 0.990809, & 0.990603,
        \\ & 0.990406, & 0.990219, & 0.990041, & 0.989873, & 0.989715, 
        \\ & 0.989567, & 0.989429, & 0.989301, & 0.989184, & 0.989077, 
        \\ & 0.988980, & 0.988894, & 0.988819, & 0.988754, & 0.988700, 
        \\ & 0.988657, & 0.988625, & 0.988603, & 0.988593, & 0.988593, 
        \\ & 0.988603, & 0.988625, & 0.988657, & 0.988700, & 0.988753, 
        \\ & 0.988817, & 0.988891, & 0.988976, & 0.989071, & 0.989176, 
        \\ & 0.989291, & 0.989416, & 0.989551, & 0.989695, & 0.989849, 
        \\ & 0.990012, & 0.990184, & 0.990365, & 0.990554, & 0.990753, 
        \\ & 0.990959, & 0.991174, & 0.991397, & 0.991627, & 0.991865, 
        \\ & 0.992110, & 0.992362, & 0.992621, & 0.992887, & 0.993158, 
        \\ & 0.993435, & 0.993719, & 0.994007, & 0.994301, & 0.994599, 
        \\ & 0.994902, & 0.995209, & 0.995520, & 0.995835, & 0.996153, 
        \\ & 0.996474, & 0.996800, & 0.997122, & 0.997452, & 0.997792, 
        \\ & 0.998104, & 0.998438, & 0.998825, & 0.999076, & 0.999378)
    \end{tabular}
\end{table}{}


%Correção e adequação do código ao problema proposto
\section{Implementação}

\end{document}

